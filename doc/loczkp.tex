\documentclass[a4paper,12pt]{article}
\usepackage{amsmath,amssymb,url}
\usepackage[unicode]{hyperref}


\author{
  \small{
  Vadym Fedyukovych
%       \url{https://platin.io/}
%       vadym at platin.io
  }
}
\title{Platin Privacy Protocol \\
\small{
  Private Location Verification with Incentivization
} \\
\large{
  Platin Technical Report
}
  %  \small{in-progress}
}


\begin{document}
\maketitle
\abstract{
  We design an interactive proof system
  for ``location is close enough'' statement,
  in the form of specification for further implementation.
}

\section{Definitions}
\label{sect-definitions}
Node proves the statement "distance is within a threshold" (less or equal)
for node coordinates $(x_n, y_n, z_n)$,
given location $(x_l, y_l, z_l)$,
and some threshold $d$ (all integers):
\begin{gather}
\label{eq-distn}
  d^2 - ((x_n - x_l)^2 + (y_n - y_l)^2 + (z_n - z_l)^2) = a_1^2 + a_2^2 + a_3^2 + a_4^2
\end{gather}
We rely on 4-squares Lagrange theorem to prove equality statement (Lipmaa).
Proofs for integer relations are possible in hidden group order setup.

\section{Proof setup}

Let $g$ be a generator of a proper group of a hidden order,
and $h$ be a group element (Pedersen commitment scheme).
We use multiplicative group of invertible residue classes modulo a composite $n$ such that
$n=pq$, $p=2p'+1$, $q=2q'+1$ and $p, q, p', q'$ primes (Idemix).

\section{Signals harvesting}
Node picks random $(r_x, r_y, r_z)$, creates commitment $(s_x, s_y, s_z)$ to it's coordinates
\begin{gather}
\label{cmt-pedr}
  s_x = g^{x_n} h^{r_x},   \qquad
  s_y = g^{y_n} h^{r_y},   \qquad
  s_z = g^{z_n} h^{r_z}
\end{gather}
and keeps coordinates-randoms pairs
$(x_n, y_n, z_n)$, $(r_x, r_y, r_z)$
private.

\subsection{Representation-based commitment}
Following U-Prove, consider commitment scheme resulting in a single group element:
\begin{gather}
\label{cmt-up}
  s_U = g_x^{x_n} g_y^{y_n} g_z^{z_n} g^{r}
\end{gather}
where $g_x, g_y, g_z$ are group elements, and $r$ is a random.
This scheme admits a proof of knowledge with the same responses required form threshold location verification.
This scheme can be extended with additional components. % like set of wireless station% name-power pairs.

\subsection{Two-level commitment}
To achieve expected properties of Merkle-tree based scheme while keeping an option
to run location proof protocols,
representation-based commitments could be leaves of Merkle tree.

\section{Proof}
\label{sect-close}
Sigma-protocol with 3 messages.
Public information is node location commitment~\eqref{cmt-up}, given location and threshold (center and radius), proof parameters.
Private information is node location and randomness to commitment.
Setup is $g, h_1 \dots h_4, g_x, g_y, g_z, g_r$
(9 group elements).

\begin{enumerate}
\item
  Prover (node) calculates~\eqref{eq-distn} $a_1 \dots a_4$ from locations and threshold
  with Rabin-Shallit algorithm~\footnote{Before R-S algorithm is implemented,
  we actually pick some $a_j$ and calculate threshold (radius) instead.},
  picks random $\alpha_j, \eta, \gamma, \beta_x, \beta_y, \beta_z, \beta_r, \rho_0, \rho_1$, %, \eta_x, \eta_y, \eta_z$,
  computes and sends initial commitments $b_0, b_1, t_a, t_n$ \\
  ($f_0$ and $f_1$ are explained at Background section)
\begin{gather}
  b_0 = g^{f_0} g_r^{\rho_0},   \qquad
  b_1 = g^{f_1} g_r^{\rho_1}    \\
  s_a = g^{\gamma} \prod_{j=1}^4 h_j^{a_j},   \qquad
  t_a = g^{\eta} \prod_{j=1}^4 h_j^{\alpha_j} ,   \qquad
  t_n = g_x^{\beta_x} g_y^{\beta_y} g_z^{\beta_z} g^{\beta_r}
%  t_y = g^{\beta_y} h^{\eta_y},   \qquad
%  t_z = g^{\beta_z} h^{\eta_z}
\end{gather}

\item
  Verifier chooses and sends his challenge $c$
\item
  Prover computes and sends responses
\begin{gather}
%  d^2 - ((x_n - x_l)^2 + (y_n - y_l)^2 + (z_n - z_l)^2) =
  X_n = c x_n + \beta_x,  \;
  Y_n = c y_n + \beta_y,  \;
  Z_n = c z_n + \beta_z,   \;
  R = c r + \beta_r   \\
  A_j = c a_j + \alpha_j, \;
  R_a = c \gamma + \eta,   \;
  R_d = c \rho_1 + \rho_0
\end{gather}

\item
  Proof verification
\begin{gather}
\label{verf-linear}
  g_x^{X_n} g_y^{Y_n} g_z^{Z_n} g^{R} s_U^{-c} = t_n, \quad
  g^{R_a} (\prod_{j=1}^4 h_j^{A_j}) s_a^{-c} = t_a \\
%
\label{verf-distn}
  g^{c^2 d^2 - ((X_n - c x_l)^2 + (Y_n - c y_l)^2 + (Z_n - c z_l)^2 ) - (A_1^2 + A_2^2 + A_3^2 + A_4^2)} g_r^{R_d} = b_1^{c} b_0
%  d^2 - ((x_n - x_l)^2 + (y_n - y_l)^2 + (z_n - z_l)^2) = a_1^2 + a_2^2 + a_3^2 + a_4^2
\end{gather}
  
\end{enumerate}

\section{Background}

Consider quadratic (degree 2 in $v$) polynomial
\begin{multline}
  f_V(v) = f_2 v^2 + f_1 v + f_0 = \\
  v^2 d^2 - (((v x_n + \beta_x) - v x_l)^2 +
           ((v y_n + \beta_y) - v y_l)^2 +
           ((v z_n + \beta_z) - v z_l)^2)  \\
        - ((v a_1 + \alpha_1)^2 +
           (v a_2 + \alpha_2)^2 +
           (v a_3 + \alpha_3)^2 +
           (v a_4 + \alpha_4)^2)
\end{multline}
This polynomial is actually linear ($f_2 = 0$, degree-one in $v$)
if, and only if
statement about distance~\eqref{eq-distn} holds for node coordinates that are hidden from verifier.
We evaluate this polynomial at a random point chosen as the challenge of verifier.
It follows, distance verification equation~\eqref{verf-distn}
only needs constant $b_0$ and degree-one $b_1^c$ components.
\begin{gather}
  f_0 = -\beta_x^2 - \beta_y^2 - \beta_z^2 - \alpha_1^2 - \alpha_2^2 - \alpha_3^2 - \alpha_4^2
\end{gather}
\begin{multline}
  f_1 = -2 (x_n - x_l) \beta_x  -2 (y_n - y_l) \beta_y  -2 (z_n - z_l) \beta_z \\
        -2 a_1 \alpha_1 -2 a_2 \alpha_2 -2 a_3 \alpha_3 -2 a_4 \alpha_4
\end{multline}
Prover calculates $b_0, b_1$ from $f_0, f_1$.

\section{Not-at-location proof}

Proving a negative location statement is a valid usecase,
that could be demonstrated with ``not at the grocery store'' scenario.
Rather that proving ``distance is smaller than''~\eqref{eq-distn},
complementary ``is larger'' proof is given. % as follows:
In the following, we only show changes required to the main protocol.
\begin{gather}
\label{eq-distn-more}
  ((x_n - x_l)^2 + (y_n - y_l)^2 + (z_n - z_l)^2) - d^2 = a_1^2 + a_2^2 + a_3^2 + a_4^2  \\
\label{verf-distn-more}
  g^{((X_n - c x_l)^2 + (Y_n - c y_l)^2 + (Z_n - c z_l)^2 ) - c^2 d^2 - (A_1^2 + A_2^2 + A_3^2 + A_4^2)} h^{R_a} = b_1^{c} b_0 
\end{gather}
\begin{multline}
\label{eq-coeff-more}
  f_V(v) = f_2 v^2 + f_1 v + f_0 = \\
  (((v x_n + \beta_x) - v x_l)^2 +
   ((v y_n + \beta_y) - v y_l)^2 +
   ((v z_n + \beta_z) - v z_l)^2)
  - v^2 d^2 \\
  - ((v a_1 + \alpha_1)^2 +
     (v a_2 + \alpha_2)^2 +
     (v a_3 + \alpha_3)^2 +
     (v a_4 + \alpha_4)^2)
\end{multline}

\section{Logical-OR threshold location}

Consider a franchise operating multiple stores,
and a usecase of proving location is ``at Starbucks'' without telling which one of $K$ known.
We define each such store with it's center $(x_k, y_k, z_k)$ and radius (size)~$d_k$, $k \in [1 .. K]$.
We elaborate basic threshold proof such that prover can produce
4-squares representation for center-size of some store $k=p$,
and pick arbitrary 4-tuples for all other stores $k \ne p$.
\begin{multline}
\label{eq-distn-or}
  \prod_{k=1}
    ((d_k^2 - ((x_n - x_k)^2 + (y_n - y_k)^2 + (z_n - z_k)^2) \\
     - (a_{1, k}^2 + a_{2, k}^2 + a_{3, k}^2 + a_{4, k}^2)) = 0
\end{multline}
Verifier is testing that polynomial $f_{KV}(v)$ is of degree at most $2K-1$, not $2K$.
\begin{multline}
  f_{KV}(v) = \sum_{j=0}^{2K} f_j v^j = \\
  \prod_{k=1}^{K} (
    v^2 d_k^2 - (((v x_n + \beta_x) - v x_k)^2 +
               ((v y_n + \beta_y) - v y_k)^2 +
               ((v z_n + \beta_z) - v z_k)^2)  \\
        - ((v a_{1,k} + \alpha_1)^2 +
           (v a_{2,k} + \alpha_2)^2 +
           (v a_{3,k} + \alpha_3)^2 +
           (v a_{4,k} + \alpha_4)^2) )
\end{multline}

\section{Private incentivization}

To engage users, verifier (map service) is giving tokens in exchange for verifying location proofs.
To keep privacy of users, proof verification and token issuance are separated with intermediate tokens.
Intermediate tokens should unlink locations that users were confirming.

Intermediate token is a Schnorr proof instance, non-interactive variant (Fiat-Shamir),
in another group of a known prime order $q$ generated by $g_i$.
Issuing (private) key $x_i$ and public key $X_i$ of map service.

A new blinded intermediate token is sent to user on each successful location verification.
Users unblind their intermediate tokens and periodically exchange them for Platin tokens.

\begin{enumerate}
\item
  Issuer (map service) chooses random $\gamma$
  and sends $\tilde w$ to Recipient
\begin{gather}
  \tilde w = g_i^{\gamma}
\end{gather}
  
\item
  Recipient (node) chooses random blinding $(\delta, \mu)$,
  produces blinded challenge~$\tilde c_i$ and sends it to Issuer
\begin{gather}
  w = \tilde w g_i^{-\delta} X_i^{\mu} \\
  \tilde c_i = H(w) + \mu  \pmod{q}
\end{gather}
\item
  Issuer produces blinded response $\tilde r_i$ and sends it to Recipient
\begin{gather}
  \tilde r_i = \tilde c_i x_i + \gamma  \pmod{q}
\end{gather}
\item
  Recipient verifies validity of response received and
  unblinds challenge and response
\begin{gather}
  \tilde c_i = H(g_i^{\tilde r_i} X_i^{- \tilde c_i}) \\
  c_i = \tilde c_i - \mu  \pmod{q} \\
  r_i = \tilde r_i - \delta  \pmod{q}
\end{gather}
  Intermediate token is $(c_i, r_i)$.
  
\item
  Intermediate token verification (while exchange for Platin tokens)
\begin{gather}
  c_i = H(g_i^{r_i} X_i^{-c_i})
% do not delete
%  w = g_i^{(\tilde c_i x + \gamma) - \delta} X_i^{- \bar c_i + \mu} = \tilde w g_i^{-\delta} X_i^{\mu}
\end{gather}
\end{enumerate}

\section{Intermediate token properties}

Both components of intermediate token
are statistically (unconditionally) independent from blinded intermediate token.

Consider an Adversary trying to distinguish two pairs of unblinded intermediate tokens $(c_1, r_1)$ and $(c_2, r_2)$,
matching them to issuing session identified with $\tilde w$.
View of Issuer is $(\tilde c_i, \tilde r_i, c_j, r_j)$  for $j=1,2$.
\begin{gather}
  g_i^{r_j} X_i^{-c_i} \stackrel{?}{=} \tilde w g_i^{-\delta_j} X_i^{\mu_j} =
                                    \tilde w g_i^{r_j - \tilde r_i} X_i^{\tilde c_i - c_j} =
                                    (\tilde w g_i^{- \tilde r_i} X_i^{\tilde c_i}) (g_i^{r_j} X_i^{- c_j})
\end{gather}
It could be seen that the only possible relation does not depend on issuing session identifier $\tilde w$,
resulting in unconditional privacy for intermediate tokens.

\section{Platin Encoding Standard 01}

With the scenario of user device (node) being close enough to a cell tower
from sections~\ref{sect-definitions}~and~\ref{sect-close},
we define encoding of coordinates, commitment, witness and proof of location.

\subsection{Non-interactive proof}
Fiat-Shamir technique could be applied to produce a challenge $c$
with proper hash function from concatenation of group elements
$t_x, t_y, t_z, b_1, b_0$ from first step of interactive protocol.
Non-interactive proof verification would be
re-creating most of group elements and
re-producing the challenge:
\begin{multline}
  \label{verf-nidistn}
  c = H(g^{X_n} h^{R_x} s_x^{-c} || g^{Y_n} h^{R_y} s_y^{-c} || g^{Z_n} h^{R_z} s_z^{-c} || b_1 || \\
        g^{c^2 d^2 - ((X_n - c x_l)^2 + (Y_n - c y_l)^2 + (Z_n - c z_l)^2 ) - (A_1^2 + A_2^2 + A_3^2 + A_4^2)} h^{R_a} b_1^{-c})
\end{multline}
Non-interactive proof is
\begin{gather}
\label{encoding-nidistn}
  (c, (X_n, Y_n, Z_n, A_1, A_2, A_3, A_4), b_1)
\end{gather}
with group element $b_1$ and 8 integers.
With this proof, multiplicative group is used, so group elements are represented as integers.

\subsection{Plaintext encoding of non-interactive proof}

Popular bignumber libraries have plaintext hex representation for bignumbers,
designed to inter-operate with each other.
Non-interactive proof is encoded as 9-lines,
with a large number on each line, in hex encoding,
in the order given by~\eqref{encoding-nidistn}.

\subsection{Plaintext encoding of location commitment}

U-Prove style commitment~\eqref{cmt-up}
is a single group element,
represented as a single-line number in hex encoding.

\subsection{Plaintext encoding of location witness}

Witness is a common term for private data held by prover
that allows to verify validity of a statement,
open a commitment instance (equation~\eqref{cmt-up}),
or run a proof protocol (section~\ref{sect-close}).

Witness is represented as four lines plaintext integer numbers;
3 lines with $x_n, y_n, z_n$ in decimal,
and a big random number $r$ in hex encoding.

Additional locations that are input to the protocol
like coordinates of cell tower $x_l, y_l, z_l$
are encoded the same as witness:
3 lines, integer numbers, decimal notation.

\subsection{Plaintext encoding of group elements}

Group is specified with two bignumbers,
composite modulo $n$ and generator $g$,
as two lines plaintext, with hex encoding.
Additional group elements required to produce commitment~\eqref{cmt-up}
are encoded on following lines the same way.


\section{Platin Encoding Standard 02}

Geo-coordinates are generally represented as latitude-longitude-altitude triple,
that could be further converted into some local x-y-z coordinate system.
This choice of coordinates apply globally,
while PES~01 better fit local (like city-scale) statements.
To commit to lat-lon-al triple, we amend two-level commitment approach
by converting GPS data into integers and
commit with the U-Prove-style scheme~\eqref{cmt-up}.
We commit to a fixed set of locations with Merkle tree,
with U-Prove commitments considered leaves of the tree.

This two-level scheme requires storage of
\begin{enumerate}
\item
  Original raw lat-lon-al data and
  scales (multipliers) used to convert them to integers;
\item
  Random integer used to produce location commitment;
\item
  Commitments (Merkle leaves) and intermediate hashes of the tree;
\item
  Merkle root as commitment to the set of location.
\end{enumerate}

Conversion from lat-lon-alt
can be done with decimal (Decimal degrees, DD) representation of angles (not degrees-minutes-seconds, DMS)
multiplied by proper power of ten and rounded.
Keeping 5 digits after the decimal point of DD representation would result in
approximately one meter (or better) precision.

Further design decisions would be taken
to account for project ideas other than location proofs,
and for implementation-level needs.
Location security policy apply for original data and commitment random input.

\section{Proving Inside a polygon}

Consider an area defined by a polygon serving as it's border,
and proof systems for ``inside'' and ``outside'' statements.
We design ``Inside'' proof as an AND of the node location $\vec l = (x_n, y_n)$ to the right side of the plane (line in 2D),
for all lines of the polygon.
We define proper side with normale vector $\vec n = (n_x, n_y)$ to the border, scalar product, and with proper distance $d$ from the origin:
\begin{gather}
  (\vec l \vec n) \ge d
\end{gather}
As with simple ``circle'' case, any non-negative integer could be represented at a sum of 4 squares,
and this inequality can be proved for all lines (indexed by $j$) of the polygon border:
\begin{gather}
  x_n n_{x,j} + y_n n_{y,j} - d_j = a_{1,j}^2 + a_{2,j}^2 + a_{3,j}^2 + a_{4,j}^2
\end{gather}
This proof can be given both for known polygon, and for hidden polygon available as a commitment while verification.
For known polygon case, challenge $c$ and responses $X_n, Y_n, A_{1,j}, A_{2,j}, A_{3,j}, A_{4,j}$
we have a replacement equation:
\begin{gather}
  g^{Y_n c n_{x,j} + Y_n c n_{y,j} - c^2 d_j - (A_{1,j}^2 + A_{2,j}^2 + A_{3,j}^2 + A_{4,j}^2)} = b_{1,j}^c b_{0,j}
\end{gather}
with $b_1, b_0$ defined at the first step of the protocol.

%\begin{enumerate}
%\item
%  Appl
%\end{enumerate}
  
%\subsection{}

\end{document}

Platin location primitive
uuid
t time
x,y,z
example (10.4)
error



Todo

%Add awarding (payment) part to the proof system.\\
``Trajectories are close enough'' proof \\
%Introduce Merkle tree -based proofs with zk-SNARK. \\
Introduce accumulator for a set of time-locations, as a feasible alternative for Merkle tree. \\
Specify ranges for random numbers.\\
Design enforcement for ``just one single location at a time''. \\
Add references (bibliography).
