\documentclass{article}
% \documentclass{llncs}
\usepackage{amsmath,amsfonts,url}

\begin{document}

\title{Private Location Verification}
\author{Lionel Wolberger, Vadym Fedyukovych}
% \institute{Platin.io}
\maketitle

\abstract{

Location based services (LBS) rely on users' disclosing their location. 
A zero knowledge procedure is proposed where a user provides a commitment that can be verified to show that the user is within a perimeter. 

The simplest perimeter is a circle with radius R. The simplest polygon is a triangle. 
The procedure is then shown to work for an arbitrary perimeter consisting of n points. 

This proof is sufficient for many LBS, and preserves the users privacy. 

\section{Introduction}

Location based services (LBS) are a multimillion dollar market.
An LBS generally relies on a user's smartphone to disclose its location data, as people carry these devices with them almost everywhere, 
use them over four hours each day on average, 
and touch them over two thousand times a day. [Ref3]

While LBS are increasingly popular and expanding into many areas of our daily lives,
there is a  parallel and counter trend towards increasing privacy. 
Privacy preserving tactics such as Data Minimization, Selective Disclosure and Progressive Trust are increasingly expected to be applied to our digital interactions. 

For example, a user may attempt to use an online service that asks him to share his location in order to prove his geolocation.
The user may hesitate, since he knows that the service may share this information with other parties without meaningful consent on his part. 

Some LBS may not need a precise location, but only an assertion that the user is within a certain area. 

This paper details a zero knowledge procedure to enable such assertions. 
The user would share a cryptographic commitment rather than his exact location. 
The commitment is validated, 
and a proof is given that Diego is within a perimeter, 
without revealing the secret of his actual location within this perimeter. 

This zero knowledge procedure is described below. 

\subsection{Our contribution}

\section{Protocol}
\subsection{Definitions}

% Commitment, interactive proof system, witness indistinguishability and zero knowledge,
% 'algebraic' interactive proofs,
% proving relations about integers,
% groups of hidden order,
% Rabin-Shallit algorithm,
% extending Schnorr protocol into higher degrees of challenge.
% Proofs and arguments, completeness and soundness, of knowledge, zero knowledge and witness indistinguishability.
% Hardness of factoring and finding group order, StrongRSA.

\subsection{Notations}

A location based service publishes a location of interest. 
The location is a geometric area. 

A simple case is that of a circle with a central point and a radius. 

The center of the circle is shared in the clear, $(x_l, y_l, z_l)$. 
The radius of interest about this point is shared as a maximum distance $d$ from $(x_l, y_l, z_l)$. 

The user's smartphone device or node is at location $(x_n, y_n, z_n )$. 
The node keeps these coordinates a secret. 
It shares a cryptographic commitment that can prove its location is within the geometry in question. 
The procedure's setup requires a group description and parameters, selection of group elements for making commitments,
an initial message, and for the interactive version a challenge and response of a Schnorr-like protocol.

We then describe a non-interactive version. In this version we replace the challenge/response by using Fiat Shamir. 

\subsection{Interactive proof}

This section presents the interactive version of the zero knowledge proof. 

The location geometry is considered as a range, and we have only to prove that the node is within the range. 
We rely on Lagrange's classic result that every non-negative integer is a sum of four squares. 
Lipmaa extended this into a non-interactive range proof. 

A mathematical group is constructed that can support an interactive query with integers. 
The order of the group is kept secret and remains unknown to the Prover.

Schnorr proof was extended into a proof systems for polynomial relations
with polynomials of higher degree in challenge
for a number of applications~\cite{F363,CF12}.

A comparable proof system for integers was introduced at~\cite{F10}.

\subsection{Proof setup}

Proof system for relations about integers is well described at Idemix documentation~\cite{Indemix}

Multiplicative group of residue classes, RSA-like modulus.

\begin{figure}[!htb]
\begin{tabular}{|p{\linewidth}|}
\hline

Common input of Prover and Verifier is
  commitment $s_U$ to node location~\eqref{cmt-up},
  airdrop location $(x_l, y_l)$,
  threshold $d^2$,
  and parameres $(g, g_x, g_y, g_r, h_j)$:
\begin{gather}
\label{cmt-up}
  s_U = g_x^{x_n} g_y^{y_n} g^{r}
\end{gather}
Private input of Prover is
  node location $(x_n, y_n)$ and location commitment randomness~$r$,
  four numbers~$\{a_j\}$ calculated with Rabin-Shallit algorithm according to~\eqref{eq-distn}.
Statement being proved is
\begin{gather}
\label{eq-distn}
  d^2 - ((x_n - x_l)^2 + (y_n - y_l)^2) = \sum_{j=1}^4 a_j^2
\end{gather}
Protocol runs as follows:
\begin{enumerate}
\item
  Prover picks random $\alpha_j, \eta, \gamma, \beta_x, \beta_y, \beta_r, \rho_0, \rho_1$,
  computes $f_0, f_1$, and sends initial commitments $b_0, b_1, t_a, t_n$:
\begin{gather}
  f_0 = -(\beta_x^2 + \beta_y^2) - \sum_{j=1}^4 \alpha_j^2,   \;
  f_1 = -2( (x_n - x_l) \beta_x  + (y_n - y_l) \beta_y) -2 \sum_{j=1}^4 a_j \alpha_j   \\
%
  t_n = g_x^{\beta_x} g_y^{\beta_y} g^{\beta_r} ,   \;
  s_a = g^{\gamma} \prod_{j=1}^4 h_j^{a_j},   \;
  t_a = g^{\eta} \prod_{j=1}^4 h_j^{\alpha_j}, \;
  b_0 = g^{f_0} g_r^{\rho_0},  \;
  b_1 = g^{f_1} g_r^{\rho_1}
\end{gather}
%
\item
  Verifier chooses and sends his challenge $c$
\item
  Prover computes and sends responses
\begin{gather}
  X_n = c x_n + \beta_x,  \;
  Y_n = c y_n + \beta_y,  \;
  R = c r + \beta_r   \\
  A_j = c a_j + \alpha_j, \;
  R_a = c \gamma + \eta,   \;
  R_d = c \rho_1 + \rho_0
\end{gather}
%
\item
  Verifier accepts if
\begin{gather}
\label{verf-linear}
  g_x^{X_n} g_y^{Y_n} g^{R} s_U^{-c} = t_n, \quad
  g^{R_a} (\prod_{j=1}^4 h_j^{A_j}) s_a^{-c} = t_a \\
\label{verf-distn}
  g^{c^2 d^2 - ((X_n - c x_l)^2 + (Y_n - c y_l)^2) - (A_1^2 + A_2^2 + A_3^2 + A_4^2)} g_r^{R_d} = b_1^{c} b_0
\end{gather}
\end{enumerate}
\\
\hline
\end{tabular}
\caption{Private location verification protocol, interactive version}
\label{fig}
\end{figure}

\subsection{Security properties}

% Completeness and Soundness.
% We reduce

\section{Implementation}

We have this protocol implemented on top of {Crypto++} library~\footnote{\url{https://cryptopp.com/}}
serving as a bignumbers backend.

Producing four-squares witness~\cite{PollackTrevino} is a work in progress, and is not a part of the protocol reported.
To facilitate larger proof-of-concept application, a temporary approximate solution was introduced producing four squares.

\section{Discussion and Conclusion}

\bibliographystyle{plain}
% \bibliographystyle{splncs}
\bibliography{private_location_verification}

%TO BE FORMATTTED
References: [ref1] Data Minimization Paper from W3C by Wolberger et al in publication
[ref3] dscout mobile touches 2016
%\appendix
\end{document}
