\documentclass{article}
% \documentclass{llncs}
\usepackage{amsmath,amsfonts,url}

\begin{document}

\title{Private Location Verification}
\author{Vadym Fedyukovych}
% \institute{Platin.io}
\maketitle

\abstract{
We present location verification protocol and implementation that allows for location privacy.
With a Schnorr-like protocol, we verify that location committed is close enough to another known location.
Protocol was implemented with {Crypto++} library.
}

\section{Introduction}

Need for location privacy.
Known results and state of the art.
SNARK-based and 'older' interactive proofs-based solutions.

\subsection{Our contribution}

\section{Protocol}
\subsection{Definitions}

% Commitment, interactive proof system, witness indistinguishability and zero knowledge,
% 'algebraic' interactive proofs,
% proving relations about integers,
% groups of hidden order,
% Rabin-Shallit algorithm,
% extending Schnorr protocol into higher degrees of challenge.
% Proofs and arguments, completeness and soundness, of knowledge, zero knowledge and witness indistinguishability.
% Hardness of factoring and finding group order, StrongRSA.

\subsection{Notations}

Airdrop location $(x_l, y_l)$ available in clear, node location $(x_n, y_n)$ hidden (committed),
acceptable maximum distance $d$ from node to airdrop,
setup: group description and parameters, group elements for making commitments,
initial message, challenge and responces of a Schnorr-like protocol.

\subsection{Interactive proof}

Interactive argument system about integers is designed
with a group of a hidden order~\cite{},
that is, order of the group is not available to the Prover.
Inequality (not far from) statement is converted into equality
with 4-squares Lagrange theorem (Lipmaa).
Schnorr proof was extended into a proof systems for polynomial relations
with polynomials of higher degree in challenge
for a number of applications~\cite{F363,CF12}.
A comparable proof system for integers was introduced at~\cite{F10}.

\subsection{Proof setup}

Proof system for relations about integers is well described at Idemix documentation~\cite{Indemix}

Multiplicative group of residue classes, RSA-like modulus.

\begin{figure}[!htb]
\begin{tabular}{|p{\linewidth}|}
\hline

Common input of Prover and Verifier is
  commitment $s_U$ to node location~\eqref{cmt-up},
  airdrop location $(x_l, y_l)$,
  threshold $d^2$,
  and parameres $(g, g_x, g_y, g_r, h_j)$:
\begin{gather}
\label{cmt-up}
  s_U = g_x^{x_n} g_y^{y_n} g^{r}
\end{gather}
Private input of Prover is
  node location $(x_n, y_n)$ and location commitment randomness~$r$,
  four numbers~$\{a_j\}$ calculated with Rabin-Shallit algorithm according to~\eqref{eq-distn}.
Statement being proved is
\begin{gather}
\label{eq-distn}
  d^2 - ((x_n - x_l)^2 + (y_n - y_l)^2) = \sum_{j=1}^4 a_j^2
\end{gather}
Protocol runs as follows:
\begin{enumerate}
\item
  Prover picks random $\alpha_j, \eta, \gamma, \beta_x, \beta_y, \beta_r, \rho_0, \rho_1$,
  computes $f_0, f_1$, and sends initial commitments $b_0, b_1, t_a, t_n$:
\begin{gather}
  f_0 = -(\beta_x^2 + \beta_y^2) - \sum_{j=1}^4 \alpha_j^2,   \;
  f_1 = -2( (x_n - x_l) \beta_x  + (y_n - y_l) \beta_y) -2 \sum_{j=1}^4 a_j \alpha_j   \\
%
  t_n = g_x^{\beta_x} g_y^{\beta_y} g^{\beta_r} ,   \;
  s_a = g^{\gamma} \prod_{j=1}^4 h_j^{a_j},   \;
  t_a = g^{\eta} \prod_{j=1}^4 h_j^{\alpha_j}, \;
  b_0 = g^{f_0} g_r^{\rho_0},  \;
  b_1 = g^{f_1} g_r^{\rho_1}
\end{gather}
%
\item
  Verifier chooses and sends his challenge $c$
\item
  Prover computes and sends responses
\begin{gather}
  X_n = c x_n + \beta_x,  \;
  Y_n = c y_n + \beta_y,  \;
  R = c r + \beta_r   \\
  A_j = c a_j + \alpha_j, \;
  R_a = c \gamma + \eta,   \;
  R_d = c \rho_1 + \rho_0
\end{gather}
%
\item
  Verifier accepts if
\begin{gather}
\label{verf-linear}
  g_x^{X_n} g_y^{Y_n} g^{R} s_U^{-c} = t_n, \quad
  g^{R_a} (\prod_{j=1}^4 h_j^{A_j}) s_a^{-c} = t_a \\
\label{verf-distn}
  g^{c^2 d^2 - ((X_n - c x_l)^2 + (Y_n - c y_l)^2) - (A_1^2 + A_2^2 + A_3^2 + A_4^2)} g_r^{R_d} = b_1^{c} b_0
\end{gather}
\end{enumerate}
\\
\hline
\end{tabular}
\caption{Private location verification protocol}
\label{fig}
\end{figure}

\subsection{Security properties}

% Completeness and Soundness.
% We reduce

\section{Implementation}

We have this protocol implemented on top of {Crypto++} library~\footnote{\url{https://cryptopp.com/}}
serving as a bignumbers backend.

Producing four-squares witness~\cite{PollackTrevino} is a work in progress, and is not a part of the protocol reported.
To facilitate larger proof-of-concept application, a temporary approximate solution was introduced producing four squares.

\section{Discussion and Conclusion}

\bibliographystyle{plain}
% \bibliographystyle{splncs}
\bibliography{private_location_verification}

%\appendix
\end{document}
