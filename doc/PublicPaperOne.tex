\documentclass{article}
% \documentclass{llncs}
\usepackage{amsmath,amsfonts,url}

\begin{document}

\title{Private Location Verification}
\author{Lionel Wolberger, Vadym Fedyukovych}
% \institute{Platin.io}
\maketitle

\abstract{
Location based services (LBS) rely on users' disclosing their location. 
A zero knowledge procedure is proposed where a user provides a commitment that can be verified to show that the user is within a perimeter. 

The simplest perimeter is a circle with radius R. The simplest polygon is a triangle. 
The procedure is then shown to work for an arbitrary perimeter consisting of n points. 

This proof is sufficient for many LBS, and preserves the users privacy. 
}
\section{Introduction}

Location based services (LBS) are a multimillion dollar market.
An LBS generally relies on a user's smartphone to disclose its location data, as people carry these devices with them almost everywhere, 
use them over four hours each day on average, 
and touch them over two thousand times a day ~\cite{dscout}

While LBS are increasingly popular and expanding into many areas of our daily lives,
there is a  parallel and counter trend towards increasing privacy. 
Privacy preserving tactics such as Data Minimization, Selective Disclosure and Progressive Trust are increasingly expected to be applied to our digital interactions. ~\cite{CCGMDSDPT}

For example, a user may attempt to use an online service that asks him to share his location in order to prove his geolocation.
The user may hesitate, since he knows that the service may share this information with other parties without meaningful consent on his part. 

Some LBS may not need a precise location, but only an assertion that the user is within a certain area. 

This paper details a zero knowledge procedure to enable such assertions. 
The user would share a cryptographic commitment rather than his exact location. 
The commitment is validated, 
and a proof is given that Diego is within a perimeter, 
without revealing the secret of his actual location within this perimeter. 

This zero knowledge procedure is described below. 

\subsection{Our contribution}

\section{Protocol}
\subsection{Definitions}

% Commitment, interactive proof system, witness indistinguishability and zero knowledge,
% 'algebraic' interactive proofs,
% proving relations about integers,
% groups of hidden order,
% Rabin-Shallit algorithm,
% extending Schnorr protocol into higher degrees of challenge.
% Proofs and arguments, completeness and soundness, of knowledge, zero knowledge and witness indistinguishability.
% Hardness of factoring and finding group order, StrongRSA.

\subsection{Notations}

A location based service publishes a location of interest. 
The location is bounded within a perimeter, or boundary. 

A simple case is that of a circle with a central point and a radius. 

The center of the circle is shared in the clear, $(x_l, y_l, z_l)$. 
The radius of interest about this point is shared as a maximum distance $d$ from $(x_l, y_l, z_l)$. 

The user's smartphone device or node is at location $(x_n, y_n, z_n )$. 
The node keeps these coordinates a secret. 
It shares a cryptographic commitment that can prove its location is within the geometry in question. 
The procedure's setup requires a group description and parameters, selection of group elements for making commitments,
an initial message, and for the interactive version a challenge and response of a Schnorr-like protocol.

We then describe a non-interactive version. In this version we replace the challenge/response by using Fiat Shamir. 

\subsection{Interactive proof}

This section presents the interactive version of the zero knowledge proof. 

The location geometry is considered as a range, and we have only to prove that the node is within the range. 
We rely on Lagrange's classic result that every non-negative integer is a sum of four squares. 
Lipmaa extended this into a non-interactive range proof. 

A mathematical group is constructed that can support an interactive query with integers. 
The order of the group is kept secret and remains unknown to the Prover.

Schnorr proof was extended into a proof systems for polynomial relations
with polynomials of higher degree in challenge
for a number of applications~\cite{F363,CF12}.

A comparable proof system for integers was introduced at~\cite{F10}.

\subsection{Proof setup}

We restrict ourselves to the use of integers, following the practice of Idemix~\cite{Idemix}.

We setup a group based on quadratic residues, a property of numbers raised to the second power (squared) as they behave in modular arithmetic. The modulus is generated based on a safe RSA key pair. This constitutes a multiplicative group of residue classes, with an RSA-like modulus.

\begin{figure}[!htb]
\begin{tabular}{|p{\linewidth}|}
\hline

Common input of Prover and Verifier is
  commitment $s_U$ to node location~\eqref{cmt-up},
  airdrop location $(x_l, y_l, z_l)$ and
  threshold $d^2$ (integers),
  parameters $(N, g, g_x, g_y, g_z, g_r, \{h_j\})$.
\begin{gather}
\label{cmt-up}
  s_U = g_x^{x_n} g_y^{y_n} g_z^{z_n} g^{r} \pmod{N}
\end{gather}
Private input of Prover is
  node location $(x_n, y_n, z_n)$ (integers) and location commitment randomness~$r$,
  four numbers~$\{a_j\}$ calculated
% with Rabin-Shallit algorithm
  according to~\eqref{eq-distn}.
\\
Statement being proved is
\begin{gather}
\label{eq-distn}
  d^2 - ((x_n - x_l)^2 + (y_n - y_l)^2) + (z_n - z_l)^2) = \sum_{j=1}^4 a_j^2
\end{gather}
Protocol runs as follows:
\begin{enumerate}
\item
  Prover picks random $\{\alpha_j\}, \eta, \gamma, \beta_x, \beta_y, \beta_z, \beta_r, \rho_0, \rho_1$,
  produces $f_0, f_1$, sends $b_0, b_1, t_a, s_a, t_n$:
\begin{gather}
  f_0 = -(\beta_x^2 + \beta_y^2 + \beta_z^2) - \sum_{j=1}^4 \alpha_j^2   \\
  f_1 = -( (x_n - x_l) \beta_x  + (y_n - y_l) \beta_y  + (z_n - z_l) \beta_z) - \sum_{j=1}^4 a_j \alpha_j   \\
%
  t_n = g_x^{\beta_x} g_y^{\beta_y} g_z^{\beta_z} g^{\beta_r} ,   \;
  s_a = g^{\gamma} \prod_{j=1}^4 h_j^{a_j},   \;
  t_a = g^{\eta} \prod_{j=1}^4 h_j^{\alpha_j} \\
  b_0 = g^{f_0} g_r^{\rho_0},  \;
  b_1 = g^{2 f_1} g_r^{\rho_1} \pmod{N}
\end{gather}
%
\item
  Verifier chooses and sends his challenge $c$
\item
  Prover produces and sends responses
\begin{gather}
  X_n = c x_n + \beta_x,  \;
  Y_n = c y_n + \beta_y,  \;
  Z_n = c z_n + \beta_z,  \;
  R = c r + \beta_r   \\
  A_j = c a_j + \alpha_j, \;
  R_a = c \gamma + \eta,   \;
  R_d = c \rho_1 + \rho_0
\end{gather}
%
\item
  Verifier accepts if
\begin{gather}
\label{verf-linear}
  g_x^{X_n} g_y^{Y_n} g_z^{Z_n} g^{R} s_U^{-c} = t_n, \quad
  g^{R_a} (\prod_{j=1}^4 h_j^{A_j}) s_a^{-c} = t_a \\
\label{verf-distn}
  g^{c^2 d^2 - ((X_n - c x_l)^2 + (Y_n - c y_l)^2 + (Z_n - c z_l)^2) - (A_1^2 + A_2^2 + A_3^2 + A_4^2)} g_r^{R_d} = b_1^{c} b_0  \pmod{N}
\end{gather}
\end{enumerate}
\\
\hline
\end{tabular}
\caption{Private location verification protocol, interactive version}
\label{ip_fig}
\end{figure}

\subsection {Non-Interactive proof}
\begin{figure}[!htb]
\begin{tabular}{|p{\linewidth}|}
\hline

Input of Prover is
  location commitment~$s_U$~\eqref{cmt-up},
  location $(x_n, y_n, z_n)$ and random $r$ to open this commitment,
  airdrop location $(x_l, y_l, z_l)$,
  threshold $d^2$,
  parameres $(N, g, g_x, g_y, g_z, g_r, h_j)$
  and public information $pubp$.
\\
Non-interactive proof is produced as follows:
\begin{enumerate}
\item
  Prover calculates $a_1 \dots a_4$ from locations and threshold,
  picks random $\{\alpha_j\}, \eta, \gamma, \beta_x, \beta_y, \beta_z, \beta_r, \rho_0, \rho_1$,
  produces $t_n, s_a, t_a, b_0, b_1$:
\begin{gather}
  t_n = g_x^{\beta_x} g_y^{\beta_y} g_z^{\beta_z} g^{\beta_r}, \;
  s_a = g^{\gamma} (\prod_{j=1}^4 h_j^{a_j}),  \;
  t_a = g^{\eta} (\prod_{j=1}^4 h_j^{\alpha_j}) \pmod{N}
\end{gather}
\begin{gather}
  \tilde f_0 = \beta_x^2 + \beta_y^2 +\beta_z^2 + \alpha_1^2 + \alpha_2^2 + \alpha_3^2 + \alpha_4^2  \\
  \tilde f_1 = (x_n - x_l) \beta_x  + (y_n - y_l) \beta_y  + (z_n - z_l) \beta_z + a_1 \alpha_1 + a_2 \alpha_2 + a_3 \alpha_3 + a_4 \alpha_4 \\
  b_0 = g^{\tilde f_0} g_r^{\rho_0},   \quad
  b_1 = g^{2 \tilde f_1} g_r^{\rho_1} \pmod{N}
\end{gather}

\item
  Prover produces his challenge with a hash function
  from text representation of commitments generated at previous step and public information:
\begin{gather}
  c = H(t_n || s_a || t_a || b_1 || b_0 || s_U || pubp)
\end{gather}

\item
  Prover produces responses:
\begin{multline}
  X_n = -c x_n + \beta_x,  \;
  Y_n = -c y_n + \beta_y,  \;
  Z_n = -c z_n + \beta_z,   \;
  R = -c r + \beta_r   \\
  A_j = -c a_j + \alpha_j, \;
  R_a = -c \gamma + \eta,   \;
  R_d = -c \rho_1 + \rho_0
\end{multline}
\end{enumerate}
Non-interactive proof is
$(c, X_n, Y_n, Z_n, R, \{A_j\}, R_a, R_d, s_a, b_1)$.
\\
Proof verification:
\begin{multline}
\label{verf-chash}
  F_d = ((X_n + c x_l)^2 + (Y_n + c y_l)^2 + (Z_n + c z_l)^2) + (A_1^2 + A_2^2 + A_3^2 + A_4^2) - c^2 d^2 \\
  H(g_x^{X_n} g_y^{Y_n} g_z^{Z_n} g^{R} s_U^{c} ||
    s_a ||
    g^{R_a} (\prod_{j=1}^4 h_j^{A_j}) s_a^{c} ||
    b_1 ||
    g^{F_d} g_r^{R_d} b_1^c ||
    s_U ||
    pubp)
  = c
\end{multline}
\\
\hline
\end{tabular}
\caption{Location proof generation and verification, non-interactive version}
\label{ni_fig}
\end{figure}

\subsection{Security properties}

% Completeness and Soundness.
% We reduce

\section{Implementation}

We have this protocol implemented on top of {Crypto++} library~\footnote{\url{https://cryptopp.com/}}
serving as a bignumbers backend.

Producing four-squares witness~\cite{PollackTrevino} is a work in progress, and is not a part of the protocol reported.
To facilitate larger proof-of-concept application, a temporary approximate solution was introduced producing four squares.

\cite{Idemix}

\section{Discussion and Conclusion}

We have shown how a node can keep its location secret, yet allow a verifier to prove that the secret location is within a perimeter.
We offer two methods to do so: interactive and non-interactive.

\bibliographystyle{plain}
% \bibliographystyle{splncs}
\bibliography{private_location_verification}

%\appendix
\end{document}
